\section*{ANEXOS}
    %%%%%%%%%%%%%%%
    %   FIGURAS   %
    %%%%%%%%%%%%%%%
    % \begin{figure}
    %     \begin{subfigure}{\linewidth}
    %         \includegraphics[width=.17\linewidth]{imgSource/gaussianas/2gaussiana-1.png}
    %         \includegraphics[width=.17\linewidth]{imgSource/gaussianas/2gaussiana-3.png}
    %         \includegraphics[width=.17\linewidth]{imgSource/gaussianas/2gaussiana-5.png}
    %         \includegraphics[width=.17\linewidth]{imgSource/gaussianas/2gaussiana-7.png}
    %         \includegraphics[width=.17\linewidth]{imgSource/gaussianas/2gaussiana-9.png}
    %         \centering
    %         \caption{Filtros gaussianos aplicados na primeira implementação.}
    %     \end{subfigure}\par\medskip
        
    %     \centering
    %     \caption{Filtros gaussianos aplicados.}
    %     \label{fig:gauss}
    % \end{figure}
    
    %%%%%%%%%%%%%%%%%
    %   EQUAÇÕES    %
    %%%%%%%%%%%%%%%%%
    % \begin{eqfloat}[h!]
    %     \begin{equation}
    %         \sqrt{h_3^2 + h_4^2}
    %         \label{eq:h3+h4}
    %     \end{equation}
    %     \caption{Fórmula para o cálculo da junção dos filtros $h_3$ e $h_4$.}
    % \end{eqfloat}